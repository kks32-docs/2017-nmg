% This template was created by H. Huegel

%\documentclass{tuhhproc-en}
\documentclass[12pt,twoside]{tuhhproc-en}
\usepackage[latin1]{inputenc}
\usepackage[T1]{fontenc}
\usepackage{times,rotating}

\usepackage[backend=biber, style=numeric, citestyle=authoryear, sorting=nty, natbib=true]{biblatex}
\bibliography{references} %Location of references.bib only for biblatex

%%%%%%%%%%%%%%%%%%%%%%%%%%%%%%%%%%%%%%%%%%%%%%%%%%%%%%%%%%%%%%%%%%%%%%%%%%
\usepackage{cleveref}
\usepackage{dsfont}
\RequirePackage{amsmath,amsfonts,amssymb}
\RequirePackage{booktabs}

\usepackage[per=slash]{siunitx} % use this package module for SI units
\usepackage{subcaption}

\selectlanguage{english}
\pagenumbering{gobble}

\begin{document}
\selectlanguage{english}

\Header{Granular column collapse on slope: Effect of permeability on the runout characteristics}{Krishna Kumar, Jean-Yves Delenne, Kenichi Soga}{1}

\Abstract{This paper investigates the effect of permeability on the runout characteristics of collapse of  granular columns on slopes in fluid. Two-dimensional sub-grain scale numerical simulations are performed to understand the flow dynamics of granular collapse in fluid. The Discrete Element (DEM) technique is coupled with the Lattice Boltzmann Method (LBM), for fluid-grain interactions, to understand the evolution of submerged granular flows. The fluid phase is simulated using Multiple-Relaxation-Time LBM (LBM-MRT) for numerical stability. In order to simulate interconnected pore space in 2D, a reduction in the radius of the grains (hydrodynamic radius) is assumed during LBM computations. The collapse of granular column in fluid is compared with the dry cases to understand the effect of fluid on the runout behaviour. A parametric analysis is performed to assess the influence of the granular characteristics (initial packing) on the evolution of flow and run-out distances for slope angle of \SI{5}{\degree}. In order to understand the effect of permeability on granular flow down a slope angle of \SI{5}{\degree}, the collapse of a granular column with an initial aspect ratio of 0.8 is simulated with different permeabilities. The hydrodynamic radius of a loosely packed granular column is varied from r = 0.7 R (high permeability), 0.75 R, 0.8 R, 0.85 R to 0.9 R (low permeability). The granular flow dynamics is investigated by analysing the effect of hydroplaning, water entrainment and viscous drag on the granular mass. The mechanism of energy dissipation, shape of the flow front, water entrainment and evolution of packing density is used to explain the difference in the flow characteristics of granular column collapse in fluid.}

\section{Introduction}
Catastrophic earth movement events, such as landslides, debris flows, rock avalanches and reservoir embankment failures, exemplify the potential consequences of an earth gravitational instability. Slope failure is a problem of high practical importance for both civil engineering structures and natural hazard management. The study described in this paper examines the stability of underwater slopes, which are caused by excess seepage or earthquakes.  They can damage offshore structures nearby and may generate a tsunami. 

In order to describe the mechanism of underwater granular flows, it is necessary to consider both the dynamics of the solid phase of granular matter and the role of the ambient fluid, which exists either inside the pores of the granular body and as free water outside the granular body~\citep{Denlinger2001,Iverson1997}. Initial acceleration plays a crucial role in underwater landslide propagation~\citep{Romano2017}, as the initial acceleration increases, there is a limited time for the landslide to deform during the acceleration phase. The initiation and propagation of submarine granular flows depend mainly on geometry (e.g. slope angle, lateral extent, etc), initial stress conditions, density, soil properties, and the quantity of the material destabilised. Although certain macroscopic models are capable of capturing simple mechanical behaviour (e.g.~\citet{Topin2011}), the complex fundamental mechanism that occurs at the grain scale, such as hydrodynamic instabilities, the formation of clusters, collapse, and transport, require further investigation in order to make better engineering assessment of the potential risk of damages against underwater slope failures for example. 

The momentum transfer between the discrete and the continuous phases of fluid saturated granular material significantly affects the dynamics of the flow~\citep{Peker2007}.
The grain-scale description of the granular material enriches the macro-scale variables. In particular, when the solid phase reaches a high-volume fraction, it is important to consider the strong heterogeneity arising from the contact forces between the grains, the drag interactions which counteract the movement of the grains, and the hydrodynamic forces that reduce the weight of the grains inducing a transition from a dense compacted to a dense suspended flow~\citep{Meruane2010}. 

The case of granular material movements in presence of an interstitial fluid at the grain-scale has been less studied. In this paper, we report the findings of the study on the granular column collapse in fluid in the inclined configuration using the coupled Lattice Boltzmann Method (LBM) and Discrete Element Method (DEM). We examined the effect of density and slope angle on the runout evolution.


The Lattice Boltzmann Method is a `micro-particle' based numerical time-stepping procedure for the solution of incompressible fluid flows. Consider a 2D incompressible fluid flow with density $\rho$ and kinematic viscosity \textit{v}, in a rectangular domain \textit{\textbf{D}}. The fluid domain is divided into a rectangular grid or lattice, with the same spacing \textit{`h'} in both the \textit{x-} and the \textit{y-}directions, as shown in~\cref{fig:D2Q9}. The present study focuses on two-dimensional problems, hence the \textit{D2Q9} momentum discretisation is adopted (see \citep{He1997} for naming convention).

\begin{figure}[htpb]
\centering
\includegraphics[width=0.45\textwidth]{figs/d2q9.pdf}
\caption[The Lattice Boltzmann discretisation and D2Q9 scheme]{The Lattice Boltzmann discretisation and D2Q9 scheme: (a) a standard LB lattice and histogram views of the discrete single particle distribution function/direction-specific densities $f_i$; (b) D2Q9 model}
\label{fig:D2Q9}
\end{figure}

The lattice Boltzmann Bhatnagar-Gross-Krook (LGBK) method is capable of simulating various hydrodynamics~\citep{Succi2001} and offers intrinsic parallelism. Although LBM is successful in modelling complex fluid systems, such as multiphase flows and suspensions in fluid, the LBM may lead to numerical instability when the dimensionless relaxation time $\tau$ is close to 0.5. The Multi-Relaxation Time Lattice Boltzmann Method (LBM-MRT) overcomes the deficiencies of linearlised single relaxation LBM-BGK, such as fixed Prandtl number (Pr=$\nu/\kappa$), where the thermal conductivity `$\kappa$' is unity~\citep{Liu2003a}. The LB-MRT model offers better numerical stability and has more degrees of freedom. In the formulation of the linear Boltzmann equation with multiple relaxation time approximation, the lattice Boltzmann equation is written as:

\begin{align}
&f_{\alpha}(\mathbf{x}+\mathbf{e}_i\Delta_t, t+ \Delta_t)-f_{\alpha}(\mathbf{x},t) \nonumber \\
&\mbox{\qquad\qquad} = -\mathbf{S}_{\alpha i}(f_i(\mathbf{x},t)-f_i^{eq}(\mathbf{x},t)
\end{align}

\noindent where \textbf{S} is collision matrix. The nine eigen values of \textbf{S} are all between 0 and 2 so as to maintain linear stability and the separation of scales, which means that the relaxation times of non-conserved quantities are much faster than the hydrodynamic time scales. The LGBK model is the special case in which the nine relaxation times are all equal and the collision matrix $\mathbf{S}=\frac{1}{\tau}\mathbf{I}$, where \textbf{I} is the identity matrix. The evolutionary progress involves two steps, advection and flux. The advection can be mapped to the momentum space by multiplying through by a transformation matrix \textbf{M} and the flux is still finished in the velocity space. The evolutionary equation of the multi-relaxation time lattice Boltzmann equation is written as:

\begin{align}
&\mathbf{f}(\mathbf{x}+\mathbf{e}_i\Delta_t, t+ \Delta_t)-\mathbf{f}(\mathbf{x},t) \nonumber \\
&\mbox{\qquad\qquad} = -M^{-1}\hat{\mathbf{S}} (\hat{\mathbf{f}}(\mathbf{x},t)-\hat{\mathbf{f}}^{eq}(\mathbf{x},t))
\end{align}

\noindent where \textbf{M} is the transformation matrix mapping a vector \textbf{f} in the discrete velocity space $\mathds{V}=\mathds{R}^b$ to a vector $\hat{\mathbf{f}}$ in the moment space $\mathds{V}=\mathds{R}^b$.

\begin{align}
&\hat{\mathbf{f}} = \mathbf{M}\mathbf{f}\\
&\mathbf{f}(\mathbf{x},t) = \left[f_0(\mathbf{x},t),f_1(\mathbf{x},t),\dots f_8(\mathbf{x},t)\right]^T
\end{align}

The collision matrix $\hat{\mathbf{S}} = MSM^{-1}$ in moment space is a diagonal matrix: $\hat{\mathbf{S}} =\mbox{diag} \left[ s_1, s_2, s_3,\dots s_9  \right]$. The transformation matrix \textbf{M} can be constructed via Gram-Schmidt orthgonalisation procedure. Through the Chapman-Enskog expansion~\citep{Du2006}, the incompressible Navier-Stokes equation can be recovered and the viscosity is given as:
\begin{align}
\nu=c_s^2\Delta t(\tau-0.5)
\end{align}


\subsection{Turbulence in Lattice Boltzmann Method}
Modelling fluids with low viscosity like water remains a challenge, necessitating very small values of \textit{h}, and/or $\tau$ very close to 0.5~\citep{He1997}. Turbulent flows are characterised by the occurrence of eddies with multiple scales in space, time and energy. In this study, the Large Eddy Simulation (LES) is adopted to solve for turbulent flow problems. The separation of scales is achieved by filtering of the Navier-Stokes equations, from which the resolved scales are directly obtained and unresolved scales are modelled by a one-parameter Smagorinski sub-grid methodology, which assumes that the Reynold's stress tensor is dependent only on the local strain rate~\citep{Smagorinsky1963}. The turbulent viscosity $\nu$ is related to the strain rate $S_{ij}$ and a filtered length scale `h' as follows:
\begin{align}
&\mathit{v}_{\mathit{t}}  =  (\mathit{S}_{c}\mathit{h})^{2}\overline{S}; \\
&\overline{S}  = \sqrt{\sum\limits_{\mathit{i,j}}{\tilde{S}_{\mathit{i,j}}\tilde{S}_{\mathit{i,j}}}}
\end{align}
where $\mathit{S}_{c}$ is the Smagorinski constant found to be close to 0.03~\citep{Yu2005}.

The effect of the unresolved scale motion is taken into account by introducing an effective collision relaxation time scale $\tau_{t}$, so that the total relaxation time $\tau_{*}$ is written as:
\begin{align}
\tau_{*}=\tau + \tau_{t}
\end{align}
where $\tau$ and $\tau_{t}$ are respectively the standard relaxation times corresponding to the true fluid viscosity \textit{v} and the turbulence viscosity $\mathit{v}_{\mathit{t}}$, defined by a sub-grid turbulence model. The new viscosity $\mathit{v}_{*}$ corresponding to $\tau_{*}$ is defined as:
\begin{align}
& \mathit{v}_{*}=\mathit{v}+\mathit{v}_{\mathit{t}}=\frac{1}{3}(\tau+\tau_{t}-\frac{1}{2})\mathit{C}^{2} \Delta \mathit{t}  \\
& \mathit{v}_{\mathit{t}}=\frac{1}{3}\tau_{\mathit{t}}\mathit{C}^{2} \Delta \textit{t}
\end{align}
The Smagorinski model is easy to implement and the Lattice Boltzmann formulation remains unchanged, except for the use of a new turbulence-related viscosity $\tau_{*}$. The component $s_1$ of the collision matrix becomes $s_1 = \frac{1}{\tau+\tau_t}$.


\section{Coupled LB-DEM model}
\subsection{General}
The Lattice Boltzmann approach has the advantage of accommodating large particle sizes and the interaction between the fluid and the moving particles can be modelled through relatively simple fluid - particle interface treatments. Further, employing the Discrete Element Method (DEM) to account for the particle/particle interaction naturally leads to a combined LB - DEM solution procedure. The Eulerian nature of the Lattice Boltzmann formulation, together with the common explicit time step scheme of both the Lattice Boltzmann and the Discrete Element, makes this coupling strategy an efficient numerical procedure for the simulation of particle-fluid systems~\citep{Cook2004}. The LB-DEM coupling system is a powerful fundamental research tool for investigating hydro-mechanical physics in porous media flow. To capture the actual physical behaviour of a fluid-particle system, the boundary condition between the fluid and the particle is modelled as a non-slip boundary condition, i.e. the fluid near the particle should have a similar velocity as the particle boundary. The solid particles inside the fluid are represented as solid lattice nodes. The discrete nature of lattice will result in stepwise representation of the surfaces~\citep{Kumar2015}. A very small lattice spacing is adopted to obtain smoother boundaries.

The smallest DEM grain in the system controls the size of the lattice. In the present study, a very fine resolution of $d_{min}/h = 10$ is adopted. That is, the smallest grain with a diameter dmin in the system is discretized into 100 lattice nodes ($10h \times 10h$). This provides a very accurate representation of the interaction between the solid and the fluid nodes. 

When combining the Discrete Element modelling of grain interactions with the lattice Boltzmann formulation, an issue arises. That is, there are now two time steps: $\Delta t$ for the fluid flow and $\Delta t_D$ for the particle movements. Since $\Delta t_D$ is normally smaller than $\Delta t$, $\Delta t_D$ is slightly reduced to a new value $\Delta t_s$ so that $\Delta t$ and $\Delta t_s$ have an integer ratio $n_s$:

\begin{align}
  \Delta t_s = & \Delta t / n_s \\
  n_s  = & [\Delta t/\Delta t_D]+1
\end{align}

This results in a subcycling time integration for the Discrete Element part.  $\Delta t$ every step of the fluid computation, $n_s$ sub-steps of integration are performed for DEM using the time step  $\Delta t_s$. The hydrodynamic force is unchanged during this sub-cycling.

\subsection{Modelling Permeability}
In DEM, the grain -- grain interaction is described based on the contact interactions. In a 3D granular assembly, the pore spaces between grains are interconnected, whereas in 2-D assembly, the grains are in contact with each other that result in a non-interconnected pore-fluid space. This results in a no flow condition in a 2-D case (see~\cref{fig:Carman}). In order to overcome this difficulty, a reduction in radius is assumed only during LBM computations (fluid and fluid--solid interaction), which is called the hydrodynamic radius. The hydrodynamic radius allows interconnected pore space through which the surrounding fluid can flow (hydrodynamic radius \textit{r} = 0.7 \textit{R} to 0.95 \textit{R}, where `\textit{R}' is the grain radius). The hydrodynamic radius is used only during the LBM computations, and has no effect on the grain--grain interactions computed using DEM.  Different values of macroscopic permeability can be obtained for any given initial packing by varying the hydrodynamic radius of the grains, without having to change the actual granular packing. This introduces a new parameter into the system. In a physical sense, a hydrodynamic radius represents the three-dimensional permeability of a granular assembly simulated as a two-dimensional geometry.


\begin{figure}[htpb]
\centering
\includegraphics[width=0.9\textwidth]{figs/Carman}
\caption{Relation between permeability and porosity for 
different hydrodynamic radius and comparison with the analytical solution.}
\label{fig:Carman}
\end{figure}

In order to understand the relation between the hydrodynamic radius and the permeability of a granular assembly, permeability tests are performed by varying the hydrodynamic radius r as 0.7 R, 0.75 R, 0.8 R, 0.85 R, 0.9 R and 0.95 R. The permeability values obtained are normalized by the square of the average grain diameter following the Carman-Kozeny equations~\citep{Yazdchi2011}. The comparison of normalised permeability from the 2D LB-DEM simulations with the Carman-Kozeny equations for spherical and cylindrical grain assembly for different porosities are presented in~\cref{fig:reduction}. It can be observed from the figure that the permeability decreases drastically as the radius is decreased from 0.7R to 0.95R. The granular assembly is almost impermeable for a hydrodynamic radius of 0.95R. The normalized permeability is found to match the qualitative trend of the Carman-Kozeny equations.

\begin{figure}[tbhp]
\centering
\includegraphics[width=0.45\textwidth]{figs/reduction}
\caption{Schematic representation of the hydrodynamic radius in LBM-DEM 
computation.}
\label{fig:reduction}
\end{figure}

\section{Granular column collapse on slopes in fluid}
\subsection{Problem definition}
In this study, a 2D polydisperse system ($d_{max}/d_{min} = 1.8$) of circular discs in fluid was used to understand the behaviour of granular flows on inclined planes. As shown in~\cref{fig:setup}, an inclined gravity direction was varied to examine the slope angle effect. The soil column was modelled using 1000 discs of density \SI{2650}{\kg\per\cubic\meter} and a contact friction angle of \SI{26}{\degree}. The aspect ratio `a' is defined as the ratio of the initial height ($H_i$) to the width ($L_i$) of the column. A granular column of aspect ratio `a' of 0.8 was used. The collapse of the column was simulated inside a fluid with a density of \SI{1000}{\kg\per\cubic\meter} and a kinematic viscosity of \SI{1e-6}{\square\meter\per\second}. The choice of a 2D geometry has the advantage of cheaper computational effort than a 3D case, making it feasible to simulate very large systems. To model 3D permeability nature of spheres as 2D discs, a reduction in radius: a  hydrodynamic radius r = 0.9R was adopted only for LBM computations, as described earlier. Dry column collapse was also performed to study the effect of hydrodynamic forces on the runout distance. The runout distance is normalised with respect to the initial width of the column. The granular column collapse was allowed to flow down a slopes of \SI{5}{\degree}. 


\begin{figure}[htpb]
\includegraphics[width=0.97\columnwidth]{figs/geometry}
\caption{Underwater granular collapse set-up}
\label{fig:setup}
\end{figure}

\section{Collapse of a dense granular column}
\Cref{fig:dense-s5-r09} shows the snapshots of the flow evolution of a dense granular column (aspect ratio 0.8), which has an initial packing density of $\Phi = 83\%$, on a \SI{5}{\degree} slope. The failure begins at the toe end of the column, and the shear-failure surface propagates into the column at an angle of about 45 - \SI{50}{\degree} The failure is due to collapse of the flank.  Force chains can be observed in the static region of the column. Once the material is destabilised, the surface of the flowing granular mass interacts with the surrounding fluid, resulting in the formation of turbulent vortices. The vortices are formed only during the horizontal acceleration phase and interact with the grains at the surface resulting in an irregular free surface. As the granular material ceases to flow, force chains restart to develop at the flow front, revealing consolidation of the granular mass with an increase in the shear resistance. The vortices formed during the collapse start to raise above the settled granular mass.

\begin{figure}[htpb]
\includegraphics[width=0.97\columnwidth]{figs/dense_s5_r09}
\caption{Evolution of a dense sand column (aspect ratio 0.8) on a slope of \SI{5}{\degree}. }
\label{fig:dense-s5-r09}
\end{figure}

The runout distance is measured by tracking the farthest particle that is still in contact with the main granular mass. The normalized runout distance is measured as $\Delta L = (L_f - L_i)/ L_i$, where $L_f$ is the runout distance at a given time and $L_i$ is the initial width of the column. The time is normalised as $t / \tau_c$, where $\tau_c$ is defined as the critical time of the dry granular column collapse to be fully mobilised on a horizontal plane ($\tau_c = \sqrt(H/g)$)~\citep{Staron2007a}.

\section{Effect of initial density - Loose versus Dense}
\citet{Topin2011} observed from CFD-DEM simulations of granular collapse on a horizontal plane in different viscous fluids that, for a given initial geometry, the run-out distance in the dry case is significantly higher than the submerged case, an observation similar to the experimental results of~\citet{Cassar2005}. For a given geometry, the initial volume fraction of the granular mass has a significant effect on the morphology of the granular deposits in fluid~\citep{Rondon2011a, Pailha2008}. In the dry case, inertia is responsible for the enhanced mobility at steeper slopes. In the submerged cases, however, the grain-inertial effects remain negligible because of the predominance of viscous effects on the granular dynamics~\citep{Topin2011}. This could explain the importance of the initial volume fraction on the runout behaviour of submerged granular columns. In this study, we examine the effect of initial packing density on the runout behaviour at \SI{5}{\degree} slope angle in both dry and submerged cases. 

In order to understand the influence of the initial packing density on the runout behaviour, a collapse of loose sand column (initial packing density $\Phi = 79\%$) was simulated at \SI{5}{\degree} slope angle and the results are compared to the dense sand column cases (initial packing density $\Phi = 83\%$), which was presented earlier. 

\subsection{Effect of permeability}

The effect of permeability on granular flow down a slope 
angle of \SI{5}{\degree} is investigated for the case of collapse of a granular column with an initial aspect ratio of 0.8 at different permeabilities. The 
hydrodynamic radius of a loosely packed granular column is varied from 
\textit{r} = 0.7 \textit{R} (high permeability), 0.75 \textit{R}, 0.8 
\textit{R}, 0.85 \textit{R} to 0.9 \textit{R} (low permeability). The run-out 
distance is found to increase with decreasing permeability of the granular 
assembly (\cref{fig:run5}). The run-out distance for high permeability 
conditions (\textit{r} = 0.7 \textit{R} -- 0.8 \textit{R}) are lower than their 
dry counterparts. Although, a decrease in permeability resulted in an increase 
in the run-out distance, no significant change in the run-out behaviour is 
observed for a hydrodynamic radii of up to 0.8 \textit{R}.

With a further decrease in permeability (\textit{r} = 0.85 \textit{R} and 0.9 
\textit{R}), the run-out 
distance in the fluid is longer than that observed in the dry 
condition. At a very low permeability (\textit{r} = 0.9 \textit{R}), the 
flowing granular mass
entrains more water at the base, which causes a reduction in the effective 
stress accompanied by a lubrication effect. This can be seen by a significant 
increase in the peak kinetic energy and the sustained duration of the peak 
energy, in comparison to the dry and the highly permeable conditions 
(\cref{fig:KE5}). However, the permeability of the granular column did not have 
an influence on the evolution of height during the flow. But, the dry granular 
column tends to collapse more than the immersed granular column due to the lack 
of viscous dissipation (\cref{fig:height5}).

\begin{figure}
\centering
\begin{subfigure}[b]{0.95\textwidth}
\centering
\includegraphics[width=0.95\columnwidth]{figs/Runout_loose_5_slope}
\caption{Evolution of run-out with time.}
\label{fig:run5}
\end{subfigure}\\
\begin{subfigure}[b]{0.95\textwidth}
\centering
\includegraphics[width=0.95\columnwidth]{figs/Height_loose_5_slope}
\caption{Evolution of height with time.}
\label{fig:height5}
\end{subfigure}
\caption{Evolution of run-out and height with time for different permeability 
(loose slope \SI{5}{\degree}).}
\label{fig:height_run_5}
\end{figure}

\begin{figure}
\centering
\begin{subfigure}[b]{0.95\textwidth}
\includegraphics[width=0.95\columnwidth]{figs/KE_loose_5_slope}
\caption{Evolution of kinetic energy with time.}
\label{fig:KE5}
\end{subfigure} \\
\begin{subfigure}[b]{0.95\textwidth}
\centering
\includegraphics[width=0.95\columnwidth]{figs/Voronoi_5}
\caption{Evolution of packing density with time.}
\label{fig:voro5}
\end{subfigure}
\caption{Evolution of kinetic energy and packing density with time for 
different permeability (loose slope \SI{5}{\degree}).}
\label{fig:voro_ke_5}
\end{figure}

\begin{figure}
\makebox[\linewidth][c]{
\begin{subfigure}[b]{0.95\textwidth}
	\centering
    \includegraphics[width=0.95\textwidth]{figs/loose_slope/LBM_660_Slope5_r07}
    \caption{r = 0.7 R}
    \label{fig:LBM_660_Slope5_r07}
\end{subfigure}
}\\
\makebox[\linewidth][c]{
\begin{subfigure}[b]{0.95\textwidth}
	\centering
    \includegraphics[width=0.95\textwidth]{figs/loose_slope/LBM_660_Slope5_r075}
    \caption{r = 0.75 R}
    \label{fig:LBM_660_Slope5_r075}
\end{subfigure}
} \\
\makebox[\linewidth][c]{
\begin{subfigure}[b]{0.95\textwidth}
	\centering
    \includegraphics[width=0.95\textwidth]{figs/loose_slope/LBM_660_Slope5_r08}
    \caption{r = 0.8 R}
    \label{fig:LBM_660_Slope5_r08}
\end{subfigure}
}\\
\makebox[\linewidth][c]{
\begin{subfigure}[b]{0.95\textwidth}
	\centering
    \includegraphics[width=0.95\textwidth]{figs/loose_slope/LBM_660_Slope5_r085}
    \caption{r = 0.85 R}
    \label{fig:LBM_660_Slope5_r085}
\end{subfigure}
}\\
\makebox[\linewidth][c]{
\begin{subfigure}[b]{0.95\textwidth}
	\centering
    \includegraphics[width=0.95\textwidth]{figs/loose_slope/LBM_660_Slope5_r09}
    \caption{r = 0.9 R}
    \label{fig:LBM_660_Slope5_r09}
\end{subfigure}
}
\caption{Evolution of the flow front at $t = 3\tau_c$ for different 
permeabilities (loose slope 5\si{\degree}).}
\label{fig:slope_loose_5}
\end{figure}

Positive pore-pressure generation at the base of the flow is observed for low 
permeability conditions. Inspection of the local packing density showed 
entrainment of water at the base of the flow, which can also be observed by the 
steep decrease in the packing density (\cref{fig:voro5}) for the very low 
permeability condition (\textit{r} = 0.9 \textit{R}). At the end of the flow 
($t \ge 10 \times \tau_c$), the excess pore-pressure dissipates and the 
granular flows, irrespective of their permeability, reach almost the same 
packing density.


\Cref{fig:Perm_Runout_loose_dense} shows the effect of permeability on the 
run-out behaviour for a dense and a loose granular column collapse on a slope 
of 5\si{\degree} and 0\si{\degree}. In both cases, the run-out distance 
increases with increase in the hydrodynamic radius (decrease in permeability). 
However in the dense case, the run-out distance observed in the fluid is 
shorter than the dry condition. Whereas in the loose condition, the run-out 
distance increases significantly at low permeabilities and results in a longer 
run-out distance in the submerged condition in comparison to the dry granular 
collapse. The comparison of loose and dense collapse on slopes of 
0\si{\degree} and 5\si{\degree} shows that the initial packing density plays a 
significant role in the case of collapse on a horizontal plane, however at a 
slope of 5\si{\degree}, the run-out distance is unaffected by the initial 
packing density at high permeability conditions. This shows that at high 
permeabilities, the viscous drag forces predominate resulting in almost the 
same run-out distance for both dense and loose conditions. However at a low 
permeability (\textit{r} = 0.9 \textit{R}), hydroplaning is observed in the 
case of loose 
granular column resulting in a substantially longer run-out distance than the 
dense granular column in submerged condition. 
%
\begin{figure}[!ht]
	\centering
\begin{subfigure}[b]{0.85\textwidth}
	\centering
    \includegraphics[width=\textwidth]{figs/Perm_Runout_a08_dense}
    \caption{Dense}
    \label{fig:Perm_Runout_a08_dense}
\end{subfigure}\\
\begin{subfigure}[b]{0.85\textwidth}
	\centering
    \includegraphics[width=\textwidth]{figs/Perm_Runout_a08_loose}
    \caption{Loose}
    \label{fig:Perm_Runout_a08_loose}
\end{subfigure}
\caption{Comparison between dry and submerged granular column for a slope angle 
of 0\si{\degree} and 5\si{\degree} on the effect of permeability on the run-out 
distance (Dense and Loose).}
\label{fig:Perm_Runout_loose_dense}
\end{figure}

\begin{figure}
\centering
\includegraphics[width=0.97\columnwidth]{figs/Perm_Runout_slope}
\caption{Effect of permeability on the run-out behaviour for different slope 
angle and the initial packing density.}
\label{fig:Perm_Runout_slope}
\end{figure}

\section{Mechanisms of granular column collapse in fluid}
The difference in the runout behaviour between the dense and loose granular columns suggests difference in runout mechanism. The collapse of a granular column involves three stages: an initiation stage characterised by a distinct failure surface, a runout phase characterised by horizontal acceleration which involves formation of eddies, and the final settlement phase. The mechanisms observed in each phase are discussed next.

\subsection{Initiation phase}
Comparing the initial evolution of runout between the loose and the dense cases ($t = 0$ to $3\tau_c$), see~\cref{fig:slope_loose_5,fig:dense-s5-r09}, the loose granular columns evolve faster than their dense counterpart. When a granular material is sheared in the submerged conditions, it generates negative pore water pressure initially due to the undrained conditions. In the undrained conditions, the dilation movement (volume expansion) of the granular assembly by shearing self-generates negative excess pore pressure inside the pores. The fluid inside the pores does not have time to seep in or out from the free water outside the column by the internally generated pressure gradient between the pore space and the outside free water. The negative excess pore pressures produce larger interparticle forces by bringing the grains together. Macroscopically this results in increase in the mean effective stress and provides large shear resistance temporarily until the negative excess pore pressure dissipates with time. Large negative pore pressures were observed for the dense cases. Overcoming the region of large negative pore pressure translates to smaller kinetic energy and eventually a shorter runout distance for the dense granular columns. 
	
The degree of dilation upon shearing is smaller in the loose granular column case than the dense case. Positive pore pressures can be observed along the failure plane. Positive pore pressures imply that the grains are pushing apart by increasing pore pressure during the initial phase of the collapse. The intergranular forces (or effective stress) are reduced and consequently the shear resistance decreases. This results in faster initiation and runout behaviour for the loose granular columns compared to the dense granular columns~\citep{Kumar2017}.


\subsection{Runout phase}
In the second phase, the initial vertical collapse motion is converted to horizontal acceleration. The flow is fully mobilised at $3\tau_c$, when the kinetic energy is at its peak. In the submerged cases, the dynamics of the granular flow are controlled by the following three factors: (a) hydroplaning, defined as the loss of friction between the flowing material and the bottom surface due to the presence of water, (b) water entrainment at the front of the flowing mass, which results in a decrease in the density of the flowing granular mass, and (c) the interaction of the surface of granular flow with the surrounding fluid resulting in formation of eddies and drag effects. The role of water on the dynamics of the dense and loose granular columns is discussed in this section.

\subsubsection{Hydroplaning}
Hydroplaning refers to the loss of friction, which occurs due to the entrapment of water between the granular mass and the slope, which results in zero effective stresses.~\Cref{fig:slope-hydroplaning} shows the effective stress of the flow front for a distance of $~15d$ for the slope angle \si{5}{\degree} case. The effective stress is computed from the contact forces acting between the grains and the bottom surface. The horizontal axis is normalised to the maximum length of $~15d$ from the flow front and the vertical axis is normalised to the initial maximum effective stress. The average effective stress at the bottom of the flowing granular mass over 5 time-steps is presented to avoid fluctuations in the stresses. The loose granular columns have zero average effective stress at the flow front ($~7.5d$ from the front). In contrast, the dense granular columns that have positive effective stress at the sliding interface, which creates frictional forces that work against the flow movement. The loose granular flow tends to entrain more water at the base of the flow front, creating a hydroplaning surface. This results in a longer runout distance. This corroborates the influence of hydroplaning on the loose granular columns which show higher sustained peak kinetic energy in comparison with the dense granular columns.

\begin{figure}[htpb]
\centering
\includegraphics[width=0.9\textwidth]{figs/slope-hydroplaning}
\caption{Effective stress at the flow front ($~15d$) between the dense and loose submerged cases for a slope angle of \si{5}{\degree} at time $t = 3\tau_c$}
\label{fig:slope-hydroplaning}
\end{figure}

\subsubsection{Water entrainment}
Comparing the snapshots of the granular flow at time $t=3\tau_c$ for the dense column cases~\cref{fig:dense-s5-r09} and the loose column cases~\cref{fig:slope_loose_5}, it can be seen that the overall shapes of the flow front are different.~\Cref{fig:slope-outline} shows the boundary of the flowing mass and its evolution with time for the slope angle \si{5}{\degree} case. It shows that the dense granular columns exhibit a more acute angle of attack (flow front) in contrast to a more parabolic profile for the loose granular columns. This difference in angle of flow front changes the ratio of amount of water entering the flowing granular mass to the water interacting with the surface of the granular mass.~\Cref{fig:water_entrainment_loose_dense} shows the horizontal velocity of the fluid and streamlines at the flow front of the dense and loose granular columns for the slope angle of \si{5}{\degree}. The density of streamlines entering the granular front is significantly higher for the loose column case, in comparison to the dense column case. This clearly shows the effect of the shape of flow front on water entrainment. More water entrainment results in increase in pore space (or decrease in solid fraction) at the flow front, which in turn reduces the interparticle shear resistance. 

\begin{figure}[htpb]
\centering
\includegraphics[width=0.9\textwidth]{figs/slope5_dense_loose_outline}
\caption{Profile outline evolution for dense (solid line) and loose (dashed line) cases for a slope of \si{5}{\degree}}
\label{fig:slope-outline}
\end{figure}

%
\begin{figure}[!ht]
	\centering
\begin{subfigure}[b]{0.85\textwidth}
	\centering
    \includegraphics[width=\textwidth]{figs/slope5_water_entrainment/slope5_dense_15d_water_entrainment}
    \caption{streamlines at the flow front dense case (contours show velocity of fluid entrainment at the flow front)}
    \label{fig:water_entrainment_a08_dense}
\end{subfigure}\\
\begin{subfigure}[b]{0.85\textwidth}
	\centering
    \includegraphics[width=\textwidth]{figs/slope5_water_entrainment/slope5_loose_15d_water_entrainment}
    \caption{streamlines at the flow front loose case (contours show velocity of fluid entrainment at the flow front)}
    \label{fig:water_entrainment_a08_loose}
\end{subfigure}
\caption{Water entrainment for a slope of \si{5}{\degree} (dense and loose cases).}
\label{fig:water_entrainment_loose_dense}
\end{figure}


\subsubsection{Turbulence and drag}
Closer observation of the surface of the flow front in the ~\cref{fig:dense-s5-r09} and the loose column cases~\cref{fig:slope_loose_5} cases shows that the number of turbulent vortices formed on the surface of the flow is significantly higher in the dense column cases than in the loose column cases. This can be attributed to the triangular profiles of the flowing mass of the dense column cases, in contrast to the more parabolic profiles observed in the loose granular column cases. This increase in the interaction between the flowing granular mass and the neighbouring fluid results in an increased momentum transfer and thus formation of turbulent vortices.~\Cref{fig:water_entrainment_a08_dense} shows higher density of streamlines at the top surface for the dense granular flow, which indicates increased drag resistance. This in turn reduces the velocity and the final runout distance. Viscous drag effect is found to reduce the acceleration up to about 30\% in underwater landslides~\citep{Watts2000}. In contrast the loose granular flowing mass experiences less drag, thereby reaching higher kinetic energy and longer runout distances. These mechanisms explain the loose granular masses in fluid spreading almost twice as long as the dense granular masses, as also observed in experiments~\citep{Iverson2000, Schaeffer2008}. 

\subsection{Settlement phase}
In both dense and loose cases, initially the packing density decreases with time during the runout initiation process ($t = 0$ to $3\tau_c$). As the body starts to slow down, the density then increases. At the end of the flow, both the dense and the loose cases reach a similar packing density, indicating that the initial density effect is erased and the granular mass reaches the critical state~\citep{Schofield1968}. As the granular mass settles, the eddies which developed during the horizontal acceleration phase starts to depart away from the granular surface. Force chains start to reappear at this phase, as individual particles re-establish contacts, and the shear resistance of the granular mass increases.


\section{Summary and Conclusions}

Two-dimensional LB-DEM simulations were performed to understand the effect of initial volume fraction on the behaviour of granular column collapse on slopes submerged in fluid. The cases were run with a granular material with relatively low permeability by adopting varying hydraulic radius. By doing so, it was possible to examine the effect of coupled interaction of pore pressure and interparticle force during different stages of granular column collapse.

LBM-DEM simulation results show that a dense granular column behavior was initially affected by the development of negative pore pressure, which resulted in increased interparticle shear resistance. This caused the delay in initiating the mass flow and reduced the kinetic energy to move laterally. During the runout phase, the flow front had an acute angle creating more vortices and drag forces along the top of the granular mass, which resisted the movement of the granular mass. This in turn resulted in shorter runout distances when compared to the dry granular column collapse cases.  

In the loose granular column collapse cases, positive pore pressures were observed inside the granular body as it sheared during the initiation phase. This allowed the body to initiate the movement faster than the dense column cases. During the runout phase, water was entrained at the flow front due to its particular parabolic shape and the packing density decreased, causing reduction in the interparticle forces at the flow front and creating some lubrication-like effect. The friction between the moving body and the bottom surface was also reduced to zero as there was no effective stress acting on the sliding plane (i.e. hydroplaning). The combined effect of positive pore pressure development, water entrainment in the front body and the hydroplaning resulted in longer runout distance compared to the dry granular column collapse cases that had the same loose packing density. An opposite behaviour was observed in the dense granular column cases, which shows the importance of initial packing density in the development of runout mechanisms.

In both dense and loose granular column cases, the final packing density was the same, indicating some critical state like behaviour, in which the initial packing memory is lost after large shearing of granular material.

\section{Acknowledgements}
The authors would like to thank Professor Farhang Radjai, LMGC, Montpellier, France for stimulating discussions regarding this work. The first author would like to thank the Cambridge Commonwealth, Overseas Trust and the Shell-Cambridge-Brazil collaboration for the financial support to pursue this research.

\selectlanguage{english}

{\renewcommand{\markboth}[2]{}% Remove header adjustment
  \printbibliography}

\section*{Authors}\small
Dr.\ Krishna Kumar\\
Department of Engineering \\
University of Cambridge \\
Cambridge, CB2 1PZ\\
Tel.: +44(0) 1223 748 589\\
e-mail: \url{kks32@cam.ac.uk}\\
Web: \url{www.cb-geo.com}\\

Prof.\ Jean-Yves Delenne\\
IATE INRA-CIRAD-Montpellier Supagro-UM2\\
University of Montpellier 2 \\
e-mail: \url{jean-yves.delenne@supagro.inra.fr}\\

Prof.\ Kenichi Soga\\
Department of Civil and Environmental Engineering \\
University of California Berkeley \\
California, USA\\
e-mail: \url{soga@berkeley.edu}\\
Web: \url{www.ce.berkeley.edu/people/faculty/soga}\\

\end{document}
