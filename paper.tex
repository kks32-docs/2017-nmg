% This template was created by H. Huegel

%\documentclass{tuhhproc-en}
\documentclass[12pt,twoside]{tuhhproc-en}
\usepackage[latin1]{inputenc}
\usepackage[T1]{fontenc}
\usepackage{times,rotating}

\selectlanguage{english}
\pagenumbering{gobble}

\begin{document}

\Header{Granular column collapse on slope: Effect of permeability on the runout characteristics}{Krishna Kumar, Jean-Yves Delenne, Kenichi Soga}{1}

\Abstract{This paper investigates the effect of permeability on the runout characteristics of collapse of  granular columns on slopes in fluid. Two-dimensional sub-grain scale numerical simulations are performed to understand the flow dynamics of granular collapse in fluid. The Discrete Element (DEM) technique is coupled with the Lattice Boltzmann Method (LBM), for fluid-grain interactions, to understand the evolution of submerged granular flows. The fluid phase is simulated using Multiple-Relaxation-Time LBM (LBM-MRT) for numerical stability. In order to simulate interconnected pore space in 2D, a reduction in the radius of the grains (hydrodynamic radius) is assumed during LBM computations. The collapse of granular column in fluid is compared with the dry cases to understand the effect of fluid on the runout behaviour. A parametric analysis is performed to assess the influence of the granular characteristics (initial packing) on the evolution of flow and run-out distances for slope angles of 5�. In order to understand the effect of permeability on granular flow down a slope angle of 5�, the collapse of a granular column with an initial aspect ratio of 0.8 is simulated with different permeabilities. The hydrodynamic radius of a loosely packed granular column is varied from r = 0.7 R (high permeability), 0.75 R, 0.8 R, 0.85 R to 0.9 R (low permeability). The granular flow dynamics is investigated by analysing the effect of hydroplaning, water entrainment and viscous drag on the granular mass. The mechanism of energy dissipation, shape of the flow front, water entrainment and evolution of packing density is used to explain the difference in the flow characteristics of granular column collapse in fluid. }

\section{Introduction}



\section*{Author}\small
Dr.\ Krishna Kumar\\
Department of Engineering \\
University of Cambridge \\
Cambridge, CB2 1PZ\\
Tel.: +44(0) 1223 748 589\\
e-mail: \url{kks32@cam.ac.uk}\\
Web: \url{www.cb-geo.com}\\

\end{document}
